\documentclass{article}\usepackage{amsmath,amssymb,amsthm,tikz,tkz-graph,color,chngpage,soul,hyperref,csquotes,graphicx,floatrow}\newcommand*{\QEDB}{\hfill\ensuremath{\square}}\newtheorem*{prop}{Proposition}\renewcommand{\theenumi}{\alph{enumi}}\usepackage[shortlabels]{enumitem}\usepackage[nobreak=true]{mdframed}\usetikzlibrary{matrix,calc}\MakeOuterQuote{"}\usepackage[margin=0.95in]{geometry} \newtheorem{theorem}{Theorem}
\usepackage{tabto}
	\NumTabs{20}
\usepackage{fancyhdr}
\usepackage{siunitx}
\usepackage{gensymb}
\pagestyle{fancy}

\headheight=40pt

\renewcommand{\headrulewidth}{6pt}
\lhead{ \Large  \fontfamily{lmdh}\selectfont
CS 280 		\tab	\tab Computer Vision 
\\Spring 2018	\tab	\tab Michael Luo, Alex Li}
\rhead{\Large 	\fontfamily{lmdh}\selectfont	Homework 3}


\begin{document}

\subsection*{1. Fundamental Matrix}
\begin{mdframed}
No, we are not directly optimizing the residual. We are optimizing over the reconstruction error, which is in the third part. The residual represents the average distance between the epipolar line of $ x_2$ and point $ x_1 $ when viewed from both cameras. The reconstruction error is the average distance between the 3D reconstructed point projected onto one of the image planes. The objective relates to the residual since we are trying to measure an average distance metric.  
\\\\In essence, we followed the guidelines of the 8-point algorithm. The first step was to normalize the points. To do so, we constructed matrixes $ T_1 $ and $ T_2 $, which were linear transformations and, upon multiplying by the matching points, normalized the points. Our matrix $ T $ has the following format, where $ \mu_x, \mu_y $ and $ \sigma_x, \sigma_y $ are the mean and std of all x and y coordinates for one camera:

\[ 
\begin{bmatrix}
    x_{norm} \\
    y_{norm} \\
    1 \\
  \end{bmatrix}
  =
\begin{bmatrix}
    \frac{1}{\sigma_{x}} & 0 & -\frac{\mu_{x}}{\sigma_{x}} \\
    0 & \frac{1}{\sigma_{y}} & -\frac{\mu_{y}}{\sigma_{y}} \\
    0 & 0 & 1 \\
\end{bmatrix}
\begin{bmatrix}
    x \\
    y \\
    1 \\
\end{bmatrix}
\]
\\After normalizing, we constructed matrix $ A $ as said in the Homework PDF and performed an SVD on $ A $, taking the last column of $ V $, since we're trying to minimize the Rayleigh's Quotient. We reshape the last column of $ V $ to the matrix $ F^* $, which is not rank 2.\\\\ Upon obtaining $ F^* $, we need to convert the matrix to rank 2, so we use Eckhart Young theorem, which just does SVD on $ F^* $ and sets the smallest singular value to 0. \\\\  After obtaining rank 2 version of $ F $, we need to denormalize the matrix, $ F \rightarrow T^{T}_{2}FT_{1} $. \\\\ For the residual error, we followed the correction on Piazza: $ d_{1->2} = \frac{x_2^TFx_1}{\|Fx_1\|} $ and $d_{2->1} = \frac{x_1^TF^Tx_2}{\|F^Tx_2\|}$. \\\\ Results for House:
\[ F = 
\begin{bmatrix}
 \num{-6.20566675e-08} & \num{1.45875690e-06} & \num{1.33031446e-04} \\
 \num{6.45620291e-06} & \num{-5.56578733e-07} & \num{-1.65983285e-02} \\
 \num{-1.04085120e-04} & \num{1.52331793e-02} & \num{-1.01024540e-02} \\
\end{bmatrix}
\]

\[ \text{Residual}= \num{2.638036824523416e-06} \]
Results for Library:
\[ F = 
\begin{bmatrix}
 \num{-4.86210796e-08} & \num{9.98237781e-07} & \num{-1.47341026e-04} \\
 \num{-5.80698246e-06} & \num{-5.65060960e-08} & \num{1.07254893e-02} \\
 \num{1.38164930e-03} & \num{-9.63586245e-03} & \num{-2.60674702e-01} \\
\end{bmatrix}
\]
\[ \text{Residual}= \num{1.0144671451144764e-05} \]

\end{mdframed}

\subsection*{2. Find Rotation Translation}
\begin{mdframed}
In total, there are four solutions. Let $ E = U \Sigma V^T$, where $ U =\begin{bmatrix} u_1 u_2 u_3 \end{bmatrix} $. Also, let 
\[ R_{90\degree} =  
\begin{bmatrix}
 0 & 1 & 0 \\
 -1 & 0 & 0 \\
 0 & 0 & 1 \\
\end{bmatrix}
\]
Then we have \[ R = UR_{90\degree}V^T, R = UR_{90\degree}^T V^T \] and \[ t = \pm u_3 \] This in total makes four possibilities. In our code, if the determinant of $ R $ ended up being $ -1 $, we just set $ R $ to be negative.

\end{mdframed}

\subsection*{3. Find 3D Points}
\begin{mdframed}


\end{mdframed}

\end{document}